\documentclass[12pt,a4paper]{report}
\usepackage{vntex}
\usepackage{extsizes}
%\usepackage[english,vietnam]{babel}
%\usepackage[utf8]{inputenc}

\usepackage[utf8]{inputenc}
%\usepackage[francais]{babel}
\usepackage{a4wide,amssymb,epsfig,latexsym,array,hhline,fancyhdr}
\usepackage{float}
\usepackage{amsmath}
\usepackage{amsthm}
\usepackage{multicol,longtable,amscd}
\usepackage{diagbox}%Make diagonal lines in tables
\usepackage{booktabs}
\usepackage{alltt}
\usepackage[framemethod=tikz]{mdframed}% For highlighting paragraph backgrounds
\usepackage{caption,subcaption}
\usepackage{logicproof}
\usepackage{lastpage}
\usepackage[lined,boxed,commentsnumbered]{algorithm2e}
\usepackage{enumerate}
\usepackage{color}
\usepackage{graphicx}							% Standard graphics package
\usepackage{array}
\usepackage{tabularx, caption}
\usepackage{multirow}
\usepackage{multicol}
\usepackage{rotating}
\usepackage{graphics}
\usepackage{geometry}
\geometry{%?i?u ch?nh gi?y cho c? file
	a4paper,%Gi?y A4
	total={170mm,257mm},%C? gi?y
	left=15mm,%L? tr?i
	right=15mm,%L? ph?i
	top=20mm,%L? tr?n
	bottom=20mm,%L? d??i
}
\usepackage{setspace}
\usepackage{epsfig}
\usepackage{tikz}
\usetikzlibrary{arrows,snakes,backgrounds}
\usepackage[unicode]{hyperref}
\hypersetup{urlcolor=blue,linkcolor=black,citecolor=black,colorlinks=true}
\begin{document}
	\begin{titlepage}
		\begin{center}
			{\scshape\large ĐẠI HỌC QUỐC GIA THÀNH PHỐ HỒ CHÍ MINH\par}
					{\scshape\LARGE TRƯỜNG ĐẠI HỌC BÁCH KHOA\par}
			{\scshape\Large KHOA KHOA HỌC - KỸ THUẬT MÁY TÍNH}
			\vspace{2cm}

		\end{center}
		\begin{figure}[h]
			\centering
			\includegraphics[scale=.2]{LogoBK.jpg}
		\end{figure}
		\vspace{2cm}
		\begin{center}
			{\scshape \LARGE \textbf{Bộ môn Kỹ thuật lập trình}\par}
			\vspace{1cm}
			\Large \textbf{Báo cáo Bài tập lớn số 2: Chương trình quản lý thư viện LIBPRO}
		\end{center}
		\vspace{2cm}
		\begin{center}
			\begin{tabular}{r l}
				GVHD:&TS. Lê Thành Sách\\
				Sinh viên:&Nguyễn Anh Khoa - 1611617\\
				&Phạm Quốc Nam - 1612128\\
				&Nguyễn Minh Khôi - 1611657\\
			\end{tabular}\\
		\end{center}
	\end{titlepage}
\newpage
\chapter{Tổng quan về LIBPRO}
LIBPRO. được sự gợi ý và thúc đẩy của các Giảng viên Môn Kỹ Thuật Lập Trình, là một phần mềm chuyên biệt quản trị các hoạt động trong các thư viện thông thường. LIBPRO tuy chưa phải phần mềm hoàn hảo nhất về các tiện ích và giao diện người dùng, nhưng đã đáp ứng được các yêu cầu cơ bản cho những người sử dụng. LIBPRO là sản phẩm của SIMple Group, gồm các thành viên Nguyễn Anh Khoa, Phạm Quốc Nam và Nguyễn Minh Khôi cùng phát triển.
\chapter{Phần mềm Quản lý Thư viện LIBPRO}
	\section{Đôi nét cơ bản về LIBPRO}
	LIBPRO là chương trình quản lý thư viện được xây dựng trên nền tảng ngôn ngữ C++ kết hợp với GUI Qt..., lưu dữ liệu người dùng, sách và yêu cầu mượn ỏ dạng file .json. LIBPRO được phát triển nhằm mục đích nghiên cứu và rèn luyện, nhiều tính năng mở rộng vẫn chưa được hoàn thiện, chỉ mới ở mặt ý tưởng.
	\section{Hướng dẫn cài đặt phần mềm}
	\section{Hướng dẫn sử dụng phần mềm}
		\subsection{Đăng nhập}
			\subsection{Đăng ký}
			Khi chưa có hồ sơ người dùng LIBPRO, chọn \textbf{Profile->Log In} và chọn vào ô chưa đăng ký để đăng ký.\\
			Các thông tin yêu cầu bắt buộc khi đăng ký là họ và tên người đăng ký, tên người dùng trong LIBPRO và số CMND (hoặc số thẻ căn cước).\\
			Sau khi đăng ký, người dùng sẽ được tạo một mật khẩu mới bất kỳ để đăng nhập vào LIBPRO, sau đó người dùng có thể chỉnh sửa lại mật khẩu.\\
			\subsection{Đăng nhập}
			Khi đã có hồ sơ người dùng LIBPRO, chọn \textbf{Profile->Log In} và nhập tên người dùng LIBPRO và mật khẩu để đăng nhập.\\
			Nếu quên mật khẩu, hãy nhấn vào ô tạo mật khẩu mới, và lúc này, hồ sơ người dùng sẽ được khóa lại, chờ khi người dùng nhận được mật khẩu mới từ QUẢN LÝ NGƯỜI DÙNG.\\
			\subsection{Đăng xuất}
			Nếu đã hoàn thành công việc trên hồ sơ của mình, người dùng có thể đăng xuất bằng cách chọn trên thanh menu \textbf{Profile->Log Out}.\\
			\subsection{Tùy chỉnh khác}
			Tạo mới tài khoản : các chức năng trong thư viện được cấp phát thông qua tài khoản của người dùng, vì vậy cần thiết tạo tài khoản cho người dùng bằng cách: trên thanh menu, chọn \textbf{Profile->New Account}.\\
			Để chuyển đổi giữa các tài khoản, chọn trên thanh menu \textbf{Profile->Choose Account}.\\
			Để chỉnh sửa các thông tin người dùng, bao gồm cả tên tài khoản, và mật khẩu người dùng, chọn \textbf{Profile->Edit->...} tùy theo nhu cầu của người dùng.\\
		\subsection{Xem thông báo}
			Có hai mục để xem là xem theo lịch sử và xem theo các thông báo chưa xem bao giờ.\\
		\subsection{Tùy chỉnh}
			\subsubsection{Chủ đề của LIBPRO} (tính năng mở rộng chưa được cập nhật)\\
			\subsubsection{Ngôn ngữ của LIBRPO} (tính năng mở rộng chưa được cập nhật)\\
		\subsection{Duyệt sách}
		Trên thanh menu, chọn \textbf{Library->View All Book} để xem toàn bộ sách trong thư viện.\\
		Nếu muốn tìm sách, chọn \textbf{Library->View All Book} và nhập tên sách người dùng muốn tìm vào thanh Search và chọn \textbf{Find} để tìm sách.\\
			\subsubsection{Sách yêu thích}
			Nếu muốn thêm sách vào mục yêu thích, chọn \textbf{Library->View All Book} và chọn \textbf{Add to Favorites} tại quyển sách mà người dùng thích.\\
			Để xem các sách yêu thích đã chọn, trên thanh menu, chọn \textbf{Library->Favorites}.\\
			Nếu không thích quyển sách nào trong mục yêu thích, người dùng nhấp vào nút \textbf{Out Favorites} để xóa sách đó khỏi mục yêu thích.\\
			\subsubsection{Mượn sách}
			Tại màn hình duyệt sách hoặc màn hình mục yêu thích hoặc màn hình tìm sách, nhấp vào nút \textbf{Add to Cart} để thêm sách vào giỏ.\\
			Nếu người dùng đã có tài khoản với chức năng READER, trên thanh menu, chọn \textbf{Role->READER->Cart} để gửi yêu cầu mượn sách đến thủ thư và chờ sự xác nhận của thủ thư.\\
		\subsection{Chức năng}
			Chọn trên thanh menu, \textbf{Role}, và lựa chọn các chức năng bên dưới. Lưu ý là chỉ những chức năng của tài khoản hiện thời mới khả dụng.\\
			\subsubsection{READER - ĐỘC GIẢ}
			\begin{enumerate}
				\item xem giở hàng: vào \textbf{READER->Cart} để xem giỏ hàng của người dùng, để đăng ký mượn sách.
				\item Photocopy: có thể photo các sách cho phép photo (tính năng mở rộng chưa được cập nhật)
			\end{enumerate}
			\subsubsection{REVIEWER - NHÀ PHÊ BÌNH}
			Có thể thêm review vào sách, (tính năng mở rộng chưa được cập nhật)\\
			\subsubsection{CD - XEM SÁCH NÓI}
			Có thể duyệt vào các sách nói, (tính năng mở rộng chưa được cập nhật)\\
			\subsubsection{MOVIE - XEM PHIM TƯ LIỆU}
			Có thể duyệt các phim tư liệu, các video học thuật, (tính năng mở rộng chưa được cập nhật)\\
			\subsubsection{EBOOK - NHÀ PHÊ BÌNH}
			Có thể duyệt các sách điện tử, (tính năng mở rộng chưa được cập nhật)\\
			\subsubsection{VIP}
			Có thể duyệt tất cả các loại sách, tư liệu và phê bình, (tính năng mở rộng chưa được cập nhật)\\
			\subsubsection{LIBRARIAN - THỦ THƯ}
				\begin{enumerate}
					\item Duyệt yêu cầu mượn sách: vào \textbf{Role->LIBRARIAN->Browse Borrow} để duyệt các yêu cầu mượn sách từ người dùng.
					\item Quản lý sách: vào \textbf{Role->LIBRARIAN->Book Management} để thêm sách, xóa sách và thay đổi các thông tin của sách.
				\end{enumerate}
			\subsubsection{USER ADMINISTRATOR - QUẢN LÝ NGƯỜI DÙNG}
				\begin{enumerate}
					\item Quản lý hồ sơ người dùng: vào \textbf{Role->USER ADMINISTRATOR->Profile Management} để thêm, xóa người dùng và tạo mật khẩu mới cho người dùng quên mật khẩu.
					\item Thêm vào danh sách đặc biệt : (tính năng mở rộng chưa được cập nhật).
					\item Khóa hồ sơ người dùng: \textbf{Role->USER ADMINISTRATOR->Lock} để khóa các hồ sơ vi phạm.
				\end{enumerate}
			\subsubsection{ACCOUNTANT - QUẢN LÝ TÀI CHÍNH}
			Quản lý chi tiêu của thư viện và của người dùng. (tính năng mở rộng chưa được cập nhật)
		\subsection{Trợ giúp}
			\begin{enumerate}
				\item Đọc hướng dẫn sử dụng thư viện.
				\item Đọc hướng dẫn sử dụng LIBPRO.
				\item Về người phát triển phần mềm.
				\item Liên hệ.
			\end{enumerate}
\chapter{Tiến trình thực hiện phần mềm}
	\section{Phân tích hoạt động của thư viện}
		\subsection{Vai trò người sử dụng}
		LIBPRO cung cấp 4 vai trò chủ yếu cho người sử dụng:
		\begin{enumerate}
			\item KHÁCH
			\item ĐỘC GIẢ
			\item THỦ THƯ
			\item QUẢN LÝ NGƯỜI DÙNG
		\end{enumerate}
			\subsubsection{KHÁCH}
				KHÁCH được định nghĩa là người được truy cập vào phần mềm LIBPRO, tuy nhiên, KHÁCH chỉ có thể xem phần trợ giúp \textbf{Help} trê thanh Menu. Số lượng KHÁCH truy cập vào phần mềm là không giới hạn
			\subsubsection{ĐỘC GIẢ}
				ĐỘC GIẢ được định nghĩa là người được truy cập vào phần mềm LIBPRO, có thể xem sách, xem thông báo, gửi yêu cầu mượn sách đến thủ thư, điều chỉnh các thông tin cá nhân, giao diện LIBPRO.
				Số lượng ĐỘC GIẢ truy cập vào phần mềm là có giới hạn, tùy theo quy định của thư viện.
			\subsubsection{THỦ THƯ}
				THỦ THƯ được định nghĩa là người được truy cập vào phần mềm LIBPRO, có khả năng quản lý sách, quản lý việc mượn và trả sách, gửi thông báo về sách mới, sách được mượn đến người dùng.
				Số lượng THỦ THƯ truy cập vào phần mềm là rất ít, do QUẢN LÝ THƯ VIỆN cấp phát.
			\subsubsection{QUẢN LÝ NGƯỜI DÙNG}
				QUẢN LÝ NGƯỜI DÙNG được định nghĩa là người được truy cập vào phần mềm LIBPRO, có khả năng quản lý hồ sơ người dùng, thêm, khóa hoặc xóa người dùng, gửi thông báo đến người dùng các thông tin liên quan đến hồ sơ người dùng.
				Số lượng QUẢN LÝ NGƯỜI DÙNG truy cập vào phần mềm rất ít, do QUANR LÝ THƯ VIỆN cấp phát.
		Ngoài ra, LIBRPO còn cung cấp thêm một số vai trò khác
		\begin{enumerate}
			\item NHÀ SƯU TẦM: có thể mua sách
			\item NHÀ PHÊ BÌNH: có thể bình luận và đánh giá sách
			\item SÁCH NÓI: truy cập vào kho sách nói
			\item TƯ LIỆU PHIM ẢNH: truy cập vào kho phim tư liệu
			\item SÁCH ĐIỆN TỬ: truy cập vào kho sách điện tử
			\item VIP: có thể xem tất cả các loại sách, tư liệu
			\itme QUẢN LÝ TÀI CHÍNH: quản lý các chi tiêu của thư viện và người dùng
		\end{enumerate}
		\subsection{Cấp độ quản lý và thực thi}
			\subsubsection{Cấp Người dùng}
			Đây là cấp lớn nhất và khái quát nhất trong môi trường LIBPRO.\\
			Một số đặc điểm của cấp Người dùng:
			\begin{enumerate}
				\item Cấp người dùng sẽ được cấp phát cho người dùng sau khi QUẢN LÝ NGƯỜI DÙNG chấp thuận yêu cầu đăng ký của người dùng.
				\item Tên gọi cấp Người dùng là duy nhất và không trùng lặp, xét cả trường hợp chữ in và chữ thường
				\item Tên gọi cấp Người dùng có độ dài không quá 32 ký tự và không chứa các ký tự đặc biệt (về các ký tự đặc biệt, xem kỹ hơn ở mục...)
				\item Một tên Người dùng có thể đăng ký nhiều tài khoản phân biệt và không trùng lặp
				\item Cấp người dùng chỉ được xóa khi người dùng gửi yêu cầu đến QUẢN LÝ NGƯỜI DÙNG hoặc do người dùng vi phạm các quy định của thư viện ở mức nào đó trở lên.
			\end{enumerate}

			\subsubsection{Cấp Tài khoản}
			Đây là cấp thứ hai sau cấp Người dùng trong môi trường LIBPRO.\\
			Một số đặc điểm của cấp Tài khoản:
			\begin{enumerate}
				\item Cấp Tài khoản sẽ có thể cấp phát cho người dùng sau khi QUẢN LÝ NGƯỜI DÙNG chấp thuận yêu cầu đăng ký của người dùng.
				\item Tên gọi cấp Tài khoản là duy nhất và không trùng lặp, xét cả trường hợp chữ in và chữ thường
				\item Tên gọi cấp Tài khoản có độ dài không quá 32 ký tự và không chứa các ký tự đặc biệt (về các ký tự đặc biệt, xem kỹ hơn ở mục...)
				\item Một tên Tài khoản có thể đăng ký nhiều vai trò phân biệt và không trùng lặp
				\item Cấp Tài khoản sẽ bị khóa nếu tài khoản đó được liệt vào danh sách đen. (Chi tiết về danh sách đen, xem mục...)
				\item Cấp tài khoản chỉ được xóa khi người dùng gửi yêu cầu đến QUẢN LÝ NGƯỜI DÙNG hoặc do người dùng vi phạm các quy định của thư viện ở mức nào đó trở lên.
			\end{enumerate}
			\subsubsection{Cấp vai trò}
			Như đã giới thiệu ở phần 2.1.1 về các vai trò người dùng, phần này xin bổ sung thêm một số quy định
			\begin{enumerate}
				\item Số vai trò trong 1 tài khoản tối đa có thể là 3
				\item Mỗi vai trò đặc trưng cho khả năng khai thác thư viện của người dùng với từng mức phí khác nhau
				\item Trong cùng một tài khoản, không thể cùng lúc thao tác ở 2 vai trò khác nhau. BUỘC phải chuyển từ vai trò A sang vai trò B để thực hiện công việc của B chứ không thể thực hiên công việc của B khi đang ở vai trò A.
			\end{enumerate}
		\subsection{Thủ tục đăng ký và đăng nhập}
		Viêc truy cập vào LIBPRO hoàn toàn chưa cần đến việc đăng ký, như đã nêu ở trên, nhưng để nâng cao mức thao tác, người sử dùng cần đăng ký người dùng và tài khoản LIBPRO.\\
			\subsubsection{Về việc Đăng ký}
			Người dùng sẽ nhấp vào ô đăng ký để thực hiên yêu cầu đăng ký của mình. Thông tin cần điền vào sẽ có những mục bắt buộc và không bắt buộc. Sau khi điền xong, người dùng sẽ nhấn nút gửi và chờ duyệt từ QUẢN LÝ NGƯỜI DÙNG. Vì thành viên của thư viện chỉ có giới hạn, nên không phải mọi người đều đăng ký thành công.\\
			Việc đăng ký thường chỉ xảy ra khi muốn tạo thêm tài khoản mới cho cấp Người dùng hiện tại. Nếu muốn đăng ký cấp Người dùng mới, phải đăng xuất khỏi cấp người dùng hiện tại để đăng ký.\\
			Về thông tin cần phải đăng ký:
			\begin{enumerate}
				\item Tên User (Là tên để khai báo trong LIBPRO)
				\item Họ và Tên người dùng (Là tên trong khai sinh)
				\item Ngày tháng năm sinh
				\item Giới tính
				\item Quốc tịch
				\item Dân tộc
				\item Số chứng minh nhân dân
				\item CMND cấp ngày
				\item Địa chỉ hiện tại
				\item Số điện thoại
				\item Công việc
				\item Cơ quan làm việc
				\item Địa chỉ cơ quan
			\end{enumerate}
			\subsubsection{Về việc Đăng nhập}
			Người dùng sẽ nhấp vào ô đăng nhập để đăng nhập vào hồ sơ người dùng của mình. Khi đăng nhập, chỉ cần điền tên người dùng và password của người dùng đó. Sau khi đăng nhập, mọi tùy chỉnh trước lần đăng xuất cuối cùng đều sẽ được giữ nguyên.\\
			Khi người dùng muốn thoát chương trình, cần thực hiện bước đăng xuất trước khi thoát.\\
			Nếu người dùng không đăng xuất trước khi thoát khỏi chương trình, ở lần mở chương trình tiếp theo, chương trình sẽ hiện lên tên tài khoản chưa đăng xuất và yêu cầu nhập password để truy cập vào tài khoản; hoặc truy cập như là KHÁCH.\\
			\subsubsection{Về các ký tự đặc biệt}
			Phần này chủ yếu dành cho việc đăt tên người dùng hoặc tên tài khoản.\\
			Các ký hiệu đặc biệt là: \!\@\# \$ \% \^\&\*\( \)\<\>\? và dấu cách
		\subsection{Quản lý sách}
			\subsubsection{Các đặc tính của sách}
			Sách trong LIBPRO được quản lý dựa trên các thông tin sau:
			\begin{enumerate}
				\item Mã ISBN
				\item Tên Đầu sách
				\item Tên tác giả (một hoặc nhiều tác giả)
				\item Nhà xuất bản
				\item Thể loại (một hoặc nhiều thể loại)
				\item Năm xuất bản
				\item Nội dung chính
			\end{enumerate}
			Sách trong thư viện sẽ được quản lý trực tiếp qua mã sách (ISBN), mỗi cuốn sách có một mã sách riêng biệt và không trùng lặp.\\
			\textbf{Thể loại:} Các thể loại sách được phân ra như sau:
			\begin{itemize}
				\item Hội họa và Nhiếp ảnh
				\item Sách Audio
				\item Tự truyện và Hồi ký
				\item Sách trên CD
				\item Tài chính và Tiền tệ
				\item Lịch
				\item Trẻ em
				\item Kinh Thánh và Sách Kitô giáo
				\item Truyện tranh
				\item Công nghệ và Máy tính
				\item Ẩm thực
				\item Sở thích và...
				\item Deals in Books
				\item Giáo dục và Đào tạo
				\item Cơ khí và Vận tải
				\item Đồng tính
				\item Sức khỏe
				\item Lịch Sử
				\item Hài hước
				\item Luật
				\item Văn học
				\item Y học
				\item Phiêu lưu mạo hiểm
				\item Làm cha mẹ
				\item Chính trị và Khoa học xã hội
				\item Tham khảo
				\item Tôn giáo và Tín ngưỡng
				\item Lãng mạn
				\item Khoa học và Toán học
				\item Khoa học viễn tưởng
				\item Kỹ năng sống
				\item Thể thao và Hoạt động Ngoài trời
				\item Lứa tuổi thiếu niên
				\item Ôn luyện các kỳ thi
				\item Sách giáo khoa
				\item Du lịch và Khám phá
			\end{itemize}
			\subsubsection{Sắp xếp sách}
			Sách được sắp xếp theo các tiêu chí giảm dần tính ưu tiên sau:
			\begin{itemize}
				\item Theo Thời điểm xuất bản (ưu tiên sách xuất bản gần đây)
				\item Theo Bảng chữ cái (A-Z hoặc Z-A)
				\item Theo Tên tác giả
				\item Theo Thể loại sách
				\item Theo Nhà xuất bản
			\end{itemize}

			\subsubsection{Nhập sách}
			Việc nhập sách được diễn ra định kỳ hoặc không định kỳ. Định kỳ 3 tháng hoặc 6 tháng sẽ có đợt sách mới về; sách mới có thể là:
			\begin{itemize}
				\item Giống sách đã có trong thư viện, việc nhập sách mới làm tăng số lượng sách chỉ của đầu sách đó
				\item Là sách hoàn toàn mới chưa có trong thư viện, việc nhập sách mới làm tăng số lượng các đầu sách
			\end{itemize}
			Việc nhập sách mới được quản lý và thông qua bởi quyết định của thủ thư.\\
			Sau khi nhập sách mới về, thủ thư cần chủ động gửi thông báo đến các độc giả còn quyền người dùng.\\

			\subsubsection{Mượn sách}
				\begin{enumerate}
				\item \textbf{Ai có thể mượn sách}\\
				Chỉ cấp vai trò ĐỘC GIẢ có quyền gửi yêu cầu mượn sách.\\
				Việc xác nhận yêu cầu mượn sách từ ĐỘC GIẢ do THỦ THƯ xử lý.\\
				\item \textbf{Có thể mượn sách trong bao lâu}\\
				Theo quy định hiện hành và phổ thông, mỗi đầu sách chỉ được mượn nhiều nhất là 2 tuần, kể từ ngày ghi trên yêu cầu mượn sách.\\
				Nếu có nhu cầu gia hạn, cấp vai trò THỦ THƯ sẽ quyết định nếu có thể gia hạn (xem phần (...)), và chỉ được gia hạn thêm 1 tuần là tối đa.\\
				\item \textbf{Một lần mượn được bao nhiêu quyển sách}\\
				Một lần mượn (tại thời điểm bất kỳ của thời gian mượn) thì một ĐỘC GIẢ chỉ có thể mượn tối đa 4 đầu sách.\\
				Tránh việc mượn hơn 1 quyển của cùng 1 đầu sách.\\
				\end{enumerate}
			\subsubsection{Trả sách}
				\begin{enumerate}
				\item \textbf{Khi nào phải trả sách}\\
					Việc trả sách được thực hiện thông qua hai quy định sau:
					\begin{enumerate}
						\item Bất kỳ quyển sách nào được mượn đều phải trả lại thư viện trong vòng 2 tuần kể từ ngày ghi trên yêu cầu mượn sách, hoặc trong vòng 3 tuần nếu được gia hạn.\\
						\item Thư viện có yêu cầu trả sách trước thời hạn vì một số lý do cụ thể.
					\end{enumerate}
					Trường hợp ĐỘC GIẢ không trả sách cho thư viện theo hai quy định trên, sẽ có mức xử phạt tương ứng (xem phần (...))\\
				\item \textbf{Tình trạng sách}\\
					Tình trạng sách là một đại lượng có tính tương đối tùy thuộc vào quy định thư viện.\\
					Nếu sách bị hư hại thì thủ thư có quyền phạt tiền người dùng, bằng cách thông báo và trừ tiền trong tài khoản của ngươi dùng.\\
					Nếu làm mất sách, thì phải xử phạt theo quy định thư viện.\\
				\end{enumerate}
		\subsection{Quản lý người dùng}
		Việc quản lý người dùng được thực hiện thông qua vai trò QUẢN LÝ NGƯỜI DÙNG.\\
			\subsubsection{Tạo cấp Người dùng}
			Khi nhận được yêu cầu đăng ký hồ sơ người dùng LIBPRO, QUẢN LÝ NGƯỜI DÙNG sẽ quyết định chấp thuận yêu cầu đăng ký hay không.\\

			\subsubsection{Cập nhật tình trạng người dùng}
			Dựa theo lịch sử đăng nhập và đăng xuất của người dùng để đánh giá và xếp loại người dùng theo các tiêu chí ()\\ %phan nay suy nghi sau

			\subsubsection{Danh sách Đỏ}
			Các tài khoản vi phạm quy định thư viện từ mức 3 đến mức 1 sẽ được liệt kê vào một danh sách gọi là danh sách Đỏ.\\

			\subsubsection{Khóa tài khoản và Danh sách Đen}
			Việc khóa tài khoản được thực hiện theo yêu cầu từ THỦ THƯ khi tài khoản này vi phạm các quy định từ mức 4 trở xuống.\\
			Đồng thời, khi bị khóa lần đầu, các tài khoản đó sẽ được liệt kê trong một danh sách, gọi là Danh sách Đen.\\
			Sau một khoảng thời gian (tùy vào thái độ của người dùng tài khoản), tài khoản từng bị khóa sẽ được mở, nhưng tên tài khoản vẫn nằm trong danh sách Đen.\\
			Các tài khoản bị liệt vào danh sách Đen sẽ chịu một số hạn chế khi sử dụng thư viện (cụ thể, xin xem phần (...)).\\

			\subsubsection{Xóa tài khoản và Xóa người dùng}
			Các tài khoản vi phạm quy định thư viện từ mức 6 trở xuống sẽ bị xóa tài khoản cùng mọi thông tin về tài khoản đó, trừ danh sách Đỏ và danh sách Đen\\
			Các người dùng vi phạm quy định thư viện từ mức 8 trở xuống sẽ bị xóa tài khoản cùng mọi thông tin về tài khoản đó, trừ danh sách Đỏ và danh sách Đen.\\
		%TÍNH NĂNG BỔ SUNG
		%tạo forum hoặc fanpage
		%hiện ra các sách liên quan đến sở thích người dùng, dùng thuật toán thống kê phân tích để xuất kết quả
	\section{Vẽ sơ đồ thiết kế dữ liệu}
	\section{Quá trình làm việc}
		\subsection{Tạo thư mục làm việc trên VisualStudio.com}
		\subsection{Sử dụng Git để quản lý và cập nhật tài liệu}
		\subsection{Hiện thực phần mềm trên CommandPrompt}
		\subsection{Thiết kế giao diện trên Qt GUI}
		\subsection{Ghép giữa dữ liệu nền và giao diện}
		\subsection{Đóng gói phần mềm}
	\section{Thực hiện tài liệu của phần mềm}
	\section{Thực hiện báo cáo về phần mềm}
\chapter{Các vấn đề nảy sinh trong tiến trình phát triển phần mềm}
	\section{Lưu dữ liệu người dùng bằng file *.txt là một trở ngại}
	\section{Việc đồng bộ hóa giữa dữ liệu và giao diện}
	\section{Thiết kế phần mềm hướng đến người dùng}
\end{document}